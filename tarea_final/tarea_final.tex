\documentclass[11pt,letterpaper]{article}
\usepackage{fullpage}
\usepackage[top=1.5cm, bottom=4.5cm, left=2cm, right=2cm]{geometry}
\usepackage{amsmath,amsthm,amsfonts,amssymb,amscd}
\usepackage{lastpage}
\usepackage{enumitem}
\usepackage{fancyhdr}
\usepackage{mathrsfs}
\usepackage{xcolor}
\usepackage{graphicx}
\usepackage{hyperref}
\usepackage{listings}
\usepackage[utf8]{inputenc}
\usepackage[spanish]{babel}

\hypersetup{%
  colorlinks=true,
  linkcolor=blue,
  linkbordercolor={0 0 1}
}

\setlength{\parindent}{0.0in}
\setlength{\parskip}{0.05in}

\lstset{
  frame=tb,
  backgroundcolor=\color{gray!10},
  inputencoding=utf8,
  literate={ñ}{{\~n}}1 {á}{{\'a}}1 {é}{{\'e}}1 {í}{{\'i}}1 {ó}{{\'o}}1 {ú}{{\'u}}1,
}
\renewcommand{\lstlistingname}{Algoritmo}

\newcommand\course{Análisis de Algoritmos}
\newcommand\hwnumber{4}
\newcommand\AStudentName{González Alvarado Raúl}
\newcommand\AStudentID{313245312}

\pagestyle{fancyplain}
\headheight 35pt
\lhead{\AStudentName\\\AStudentID}
\rhead{\course \\ \today}
\chead{\textbf{\Large Tarea \hwnumber}}
\lfoot{}
\cfoot{}
\rfoot{\small\thepage}
\headsep 1cm

\newcommand\respuesta{\textbf{\textit{Respuesta: }}}
\newcommand{\af}{\textit{Afirmación }}

\begin{document}
  \begin{enumerate}[leftmargin=*]
    \item Queremos ordenar una lista $S$ de $n$ enteros que contiene muchos elementos duplicados.
    Supongamos que los elementos de $S$ sólo contienen $O(\log n)$ valores distintos.

    \begin{enumerate}[label=\alph*)]
        \item Encuentre un algoritmo que toma a lo más $O(n \log (\log n))$ tiempo para ordenar $S$.
        
        \respuesta
        
        \item ¿Por qué no viola la cota inferior de $O(n \log n)$ para el problema de ordenación.
        
        \respuesta
    \end{enumerate}

    \item Supongamos que tenemos que ordenar una lista $L$ de $n$ enteros cuyos valores están entre $1$ y $m$.
    Pruebe que si $m$ es $O(n)$ entonces los elementos de $L$ pueden ser ordenados en tiempo lineal.
    ¿Qué pasa si $m$ es de $O(n^2)$? ¿Se puede realizar en tiempo lineal? ¿Por qué?

    \begin{enumerate}[label=\alph*)]
        \item $O(n)$
        
        \respuesta

        \item $O(n^2)$
        
        \respuesta
    \end{enumerate}

    \item Pruebe que el segundo elemento más chico de una lista de $n$ elementos distintos puede encontrarse con $n + \lceil \log n \rceil - 2$ comparaciones.
    
    \respuesta

    \item Supongamos que tenemos dos listas ordenadas $S$ y $T$, cada una con $n$ elementos.
    
    \begin{enumerate}[label=\alph*)]
        \item Encuentre un algoritmo que en $\Theta (\log^2 n)$ encuentre el $n$-ésimo elemento de $S \cup T$ (no calcule $S \cup T$, esto toma $O(n)$).
        
        \respuesta

        \item De una cota inferior sobre la complejidad de este problema, o encuentre un algoritmo más eficiente.
        
        \respuesta
    \end{enumerate}

    \item Decimos que un arreglo $A[1...n]$ está $k$-ordenado si puede ser dividio en $k$ bloques, cada uno de tamaño $n / k$, tal que los elementos de cada bloque son más grandes que los elementos de de los bloques anteriores, y más pequeños que los elementos en bloques siguientes.
    Los elementos dentro de cada bloque no necesitan estar ordenados.
    
    \begin{enumerate}[label=\alph*)]
        \item Describe un algoritmo que $k$-ordene un arreglo cualquiera en tiempo $O(n \log k)$.
        
        \respuesta
        
        \item Prueba que cualquier algoritmo de $k$-ordenación basado en comparaciones requiere $\Omega (n \log k)$ comparaciones en el peor de los casos.
        
        \respuesta
        
        \item Describe un algoritmo que ordene completamente un arreglo que ya esté $k$-ordenado en tiempo $O(n \log (n /K))$.
        
        \respuesta
    \end{enumerate}

    \item Diseña un algoritmo eficiente que tome un arreglo $A$ de $n$ enteros positivos y obtenga el número de parejas de indices invertidos, donde una pareja de indices $(i,j)$ son invertidos si $i < j$ y $A[i] > A[j]$.
    
    \respuesta

    \item En clase se mostró un algoritmo para encontrar el $k$-ésimo elemento de un arreglo $S$, partiéndolo en grupos de tamaño 5.
    De manera similar, encuentra el $k$-ésimo elemento de $S$ partiéndolo en grupos de 3 y 7.
    ¿En ambos casos se conserva la linealidad del algoritmo? Explique.

    \respuesta

    \item Dados dos árboles generadores $T$ y $R$ de una gráfica $G = (V, E)$, mostrar como encontrar la secuencia más corta de árboles generadores $T_0, T_1, ..., T_k$ tal que $T_0 = T$, $T_k = R$ y cada árbol $T_i$ difiere del anterior $T_{i-1}$ en añadir un solo vértice y eliminar otro.
    
    \respuesta

    \item Considera un tablero de ajedrez $B$ de tamaño $n \times n$ de cuadrados alternantes blancos y negros.
    Y suponemos que $n$ es par.
    Queremos cubrir todo el tablero con piezas rectangulares de dominó de tamaño $2 \times 1$.

    \begin{enumerate}[label=\alph*)]
        \item Muestra como cubrir el tablero con $\frac {n \times n} 2$ dominos.
        
        \respuesta

        \item Remueve la casilal de arriba a la izquierda y la de abajo a la derecha de $B$.
        Muestra que no se puede cubrir el tablero restante con $\frac {n \times n} 2 - 1$ dominós.

        \respuesta

        \item Remueve un cuadrado negro arbitrario y uno blanco de $B$.
        Muestra que el resto del tablero puede ser cubierto con $\frac {n \times n} 2 - 1$ dominós.

        \respuesta
    \end{enumerate}

    \item Dada una secuencia de $n$ valores $x_1, x_2, ..., x_n$ y queremos encontrar rápidamente respuesta a preguntas de la forma: dados  $i$ y $j$, encontrar el valor más pequeño entre $x_i, ... x_j$.
    
    \begin{enumerate}[label=\alph*)]
        \item Diseña una estructura de datos que use espacio $O(n^2)$ y conteste a las preguntas en tiempo $O(n)$.
        
        \respuesta

        \item Diseña una estructura de datos que use espacio $O(n)$ y conteste a las preguntas en tiempo $O(\log n)$.
        
        \respuesta
    \end{enumerate}

    \item Un pescador está sobre un océano rectangular. El valor del pez en el punto $(i, j)$ está dado por un arreglo $A$ de dimensión 2 y tamaño $n \times m$.
    Diseña un algoritmo que calcule el mácimo valor de pescado que un pescador puede atrapar en un camino desde la esquina superior izquierda a la esquina inferior derecha.
    El pescador solo puede moverse hacia abajo o hacia la derecha.

    \respuesta

    \item El intervalo común más largo de dos sucesiones $X$ y $Y$, es un conjunto de elementos consecutivos de $X$ y $Y$ más largo que aparece en ambas sucesiones (no confundir con subsucesión común más larga).
    Sean $n = |X|$ y $m = |Y|$. Encuentre un algorimo que en $\Theta (nm)$ encuentre el intervalo común más largo de $X$ y $Y$.

    \respuesta

    \item Sean tres cadenas de caracteres $X$, $Y$ y $Z$, con $|X| = n$, $|Y| = m$ y $|Z| = n + m$.
    Diremos que $Z$ es un \textit{shuffle} de $X$ y $Y$ si $Z$ puede ser formado por caracteres intercalados de $X$ y $Y$ manteniendo el orden de izquierda a derecha de cada cadena.

    \begin{enumerate}[label=\alph*)]
        \item Muestra que \textit{cchocoholaptes} es un \textit{shuffle} de \textit{chocolate} y \textit{chips}, pero \textit{chocochilatspe} no lo es.
        
        \respuesta

        \item Diseña un algoritmo de programación dinámica eficiente que determine si $Z$ es un \textit{shuffle} de $X$ y $Y$.
        
        \respuesta
    \end{enumerate}

    \item Supongamos que todos los vértices de una gráfica tienen peso distinto (es decir, ningun par de vértices tiene el mismo peso).
    ¿El camino entre un par de vértices en un árbol generador de peso mínimo es necesariamente el camino más corto entre esos dos vértices en la gráfica original?
    Dar una prueba o un contraejemplo.

    \respuesta

    \item Construya el arbol de Huffman para codificar el siguiente texto:
    
    \textit{El azote, hijo mío, se inventó para castigar afrontando al racional y para avivar la pereza del bruto que carece de razón; preo no para el niño decente y de vergüenza que sabe lo que le importa hacer y lo que nunca debe ejecutar, no amedrentado por el rigor del castigo, sino obligado por la persuación de la doctrina y el convencimiento de su propio interés.}

    \respuesta

    \item Mientras caminas por la playa encuentras un cofre de tesoros.
    El cofre contiene $n$ tesoros con pesos $w_1, ..., w_n$ y valores $v_1, ..., v_n$.
    Desafortunadamente sólo tienes una mochila que solo tiene capacidad de carga $M$.
    Afortunadamente los tesoros se pueden rompeer si es necesario.
    Por ejemplo, la tercera parte de un tesoro $i$ tiene peso $\frac {w_i} 3$ y valor $\frac {v_i} 3$.

    \begin{enumerate}[label=\alph*)]
        \item Describe un algoritmo voraz de tiempo $\Theta (n \log n)$ que resuelva este problema.
        
        \respuesta

        \item Prueba que tu algoritmo obtiene la solución correcta.
        
        \respuesta

        \item Mejora el tiempo de ejecución de tu algoritmo a $\Theta (n)$.
        
        \respuesta
    \end{enumerate}
    
    \item Sea $S$ un conjunto de $n$ puntos en el plano en posición general, tales que $\forall {(x_i, y_i)} \in S$ se tiene que $x_i, y_i \in \mathbb{N}$ y $x_i, y_i \in [0, ..., n^2]$.
    Describe un algorimo que encuentre el cierre convexo de $S$ en tiempo $O(n)$.

    \respuesta

    \item Un árbol generador mínimo Euclideano (EMST) de un conjunto $P$ de puntos en el plano es un árbol aristas de longitud mínima que conecta todos los puntos.
    Los EMST son interesantes en aplicaciones donde queremos conectar sitios en un ambiente plano con lineas de comunicación, caminos, vias de tren, etc.

    \begin{enumerate}[label=\alph*)]
        \item Prueba que el conjunto de vértices de la triangulación de Delaunay de $P$ contiene un EMST para $P$.
        
        \respuesta

        \item Usa el resultado anterior pra dar un algorimo $O (n \log n)$ para computar un EMST para $P$.
        
        \respuesta

    \end{enumerate}

    \item Sea $P$ un conjunto de $n$ puntos en el plano.
    Muestra un algorimo de tiempo $O (n \log n)$ para encontrar para cada punto $p \in P$ el punto en $P$ más cercano a $p$. 
    
    \respuesta

  \end{enumerate}
\end{document}